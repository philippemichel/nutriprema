% Options for packages loaded elsewhere
\PassOptionsToPackage{unicode}{hyperref}
\PassOptionsToPackage{hyphens}{url}
\PassOptionsToPackage{dvipsnames,svgnames,x11names}{xcolor}
%
\documentclass[
  a4paperpaper,
  french]{scrartcl}

\usepackage{amsmath,amssymb}
\usepackage{lmodern}
\usepackage{iftex}
\ifPDFTeX
  \usepackage[T1]{fontenc}
  \usepackage[utf8]{inputenc}
  \usepackage{textcomp} % provide euro and other symbols
\else % if luatex or xetex
  \usepackage{unicode-math}
  \defaultfontfeatures{Scale=MatchLowercase}
  \defaultfontfeatures[\rmfamily]{Ligatures=TeX,Scale=1}
  \setmainfont[Numbers=OldStyle,Ligatures=TeX]{Faune}
  \setsansfont[Ligatures=TeX]{Myriad Pro}
\fi
% Use upquote if available, for straight quotes in verbatim environments
\IfFileExists{upquote.sty}{\usepackage{upquote}}{}
\IfFileExists{microtype.sty}{% use microtype if available
  \usepackage[]{microtype}
  \UseMicrotypeSet[protrusion]{basicmath} % disable protrusion for tt fonts
}{}
\makeatletter
\@ifundefined{KOMAClassName}{% if non-KOMA class
  \IfFileExists{parskip.sty}{%
    \usepackage{parskip}
  }{% else
    \setlength{\parindent}{0pt}
    \setlength{\parskip}{6pt plus 2pt minus 1pt}}
}{% if KOMA class
  \KOMAoptions{parskip=half}}
\makeatother
\usepackage{xcolor}
\setlength{\emergencystretch}{3em} % prevent overfull lines
\setcounter{secnumdepth}{5}
% Make \paragraph and \subparagraph free-standing
\ifx\paragraph\undefined\else
  \let\oldparagraph\paragraph
  \renewcommand{\paragraph}[1]{\oldparagraph{#1}\mbox{}}
\fi
\ifx\subparagraph\undefined\else
  \let\oldsubparagraph\subparagraph
  \renewcommand{\subparagraph}[1]{\oldsubparagraph{#1}\mbox{}}
\fi


\providecommand{\tightlist}{%
  \setlength{\itemsep}{0pt}\setlength{\parskip}{0pt}}\usepackage{longtable,booktabs,array}
\usepackage{calc} % for calculating minipage widths
% Correct order of tables after \paragraph or \subparagraph
\usepackage{etoolbox}
\makeatletter
\patchcmd\longtable{\par}{\if@noskipsec\mbox{}\fi\par}{}{}
\makeatother
% Allow footnotes in longtable head/foot
\IfFileExists{footnotehyper.sty}{\usepackage{footnotehyper}}{\usepackage{footnote}}
\makesavenoteenv{longtable}
\usepackage{graphicx}
\makeatletter
\def\maxwidth{\ifdim\Gin@nat@width>\linewidth\linewidth\else\Gin@nat@width\fi}
\def\maxheight{\ifdim\Gin@nat@height>\textheight\textheight\else\Gin@nat@height\fi}
\makeatother
% Scale images if necessary, so that they will not overflow the page
% margins by default, and it is still possible to overwrite the defaults
% using explicit options in \includegraphics[width, height, ...]{}
\setkeys{Gin}{width=\maxwidth,height=\maxheight,keepaspectratio}
% Set default figure placement to htbp
\makeatletter
\def\fps@figure{htbp}
\makeatother

\usepackage{booktabs}
\usepackage{longtable}
\usepackage{array}
\usepackage{multirow}
\usepackage{wrapfig}
\usepackage{float}
\usepackage{colortbl}
\usepackage{pdflscape}
\usepackage{tabu}
\usepackage{threeparttable}
\usepackage{threeparttablex}
\usepackage[normalem]{ulem}
\usepackage{makecell}
\usepackage{xcolor}
\usepackage[output-decimal-marker={,}, mode = text]{siunitx}
\KOMAoption{captions}{tableheading}
\makeatletter
\makeatother
\makeatletter
\makeatother
\makeatletter
\@ifpackageloaded{caption}{}{\usepackage{caption}}
\AtBeginDocument{%
\ifdefined\contentsname
  \renewcommand*\contentsname{Table des matières}
\else
  \newcommand\contentsname{Table des matières}
\fi
\ifdefined\listfigurename
  \renewcommand*\listfigurename{Liste des Figures}
\else
  \newcommand\listfigurename{Liste des Figures}
\fi
\ifdefined\listtablename
  \renewcommand*\listtablename{Liste des Tables}
\else
  \newcommand\listtablename{Liste des Tables}
\fi
\ifdefined\figurename
  \renewcommand*\figurename{Figure}
\else
  \newcommand\figurename{Figure}
\fi
\ifdefined\tablename
  \renewcommand*\tablename{Table}
\else
  \newcommand\tablename{Table}
\fi
}
\@ifpackageloaded{float}{}{\usepackage{float}}
\floatstyle{ruled}
\@ifundefined{c@chapter}{\newfloat{codelisting}{h}{lop}}{\newfloat{codelisting}{h}{lop}[chapter]}
\floatname{codelisting}{Listing}
\newcommand*\listoflistings{\listof{codelisting}{Liste des Listings}}
\makeatother
\makeatletter
\@ifpackageloaded{caption}{}{\usepackage{caption}}
\@ifpackageloaded{subcaption}{}{\usepackage{subcaption}}
\makeatother
\makeatletter
\@ifpackageloaded{tcolorbox}{}{\usepackage[many]{tcolorbox}}
\makeatother
\makeatletter
\@ifundefined{shadecolor}{\definecolor{shadecolor}{rgb}{.97, .97, .97}}
\makeatother
\makeatletter
\makeatother
\ifLuaTeX
\usepackage[bidi=basic]{babel}
\else
\usepackage[bidi=default]{babel}
\fi
\babelprovide[main,import]{french}
% get rid of language-specific shorthands (see #6817):
\let\LanguageShortHands\languageshorthands
\def\languageshorthands#1{}
\ifLuaTeX
  \usepackage{selnolig}  % disable illegal ligatures
\fi
\usepackage[]{natbib}
\bibliographystyle{plainnat}
\IfFileExists{bookmark.sty}{\usepackage{bookmark}}{\usepackage{hyperref}}
\IfFileExists{xurl.sty}{\usepackage{xurl}}{} % add URL line breaks if available
\urlstyle{same} % disable monospaced font for URLs
\hypersetup{
  pdftitle={NUTRI'PRÉMA},
  pdfauthor={Philippe MICHEL},
  pdflang={fr-FR},
  colorlinks=true,
  linkcolor={blue},
  filecolor={Maroon},
  citecolor={Blue},
  urlcolor={Blue},
  pdfcreator={LaTeX via pandoc}}

\title{NUTRI'PRÉMA\thanks{Paul GALISSON, Dr Suzanne BORRHOMEE -
Dermatologie}}
\usepackage{etoolbox}
\makeatletter
\providecommand{\subtitle}[1]{% add subtitle to \maketitle
  \apptocmd{\@title}{\par {\large #1 \par}}{}{}
}
\makeatother
\subtitle{Plan d'analyse statistique}
\author{Philippe MICHEL}
\date{27/01/2023}

\begin{document}
\maketitle
\ifdefined\Shaded\renewenvironment{Shaded}{\begin{tcolorbox}[sharp corners, boxrule=0pt, enhanced, interior hidden, breakable, borderline west={3pt}{0pt}{shadecolor}, frame hidden]}{\end{tcolorbox}}\fi

Ce document ne concerne que l'analyse statistique des données.

Il s'agit d'une étude prospective non interventionnelle mono-centrique
non randomisée. Le risque \(\alpha\) retenu est de 0,05 \& la puissance
de 0,8.

Sauf indication contraire les données numériques seront présentées par
leur médiane avec les quartiles é comparées par des tests du t de
Student si les conditions de normalité sont remplies, sinon par des
tests de Wilcoxon. Les données catégorielles seront présentées par par
le nombre \emph{\&} le pourcentage avec son intervalle de confiance
calculé par bootstrap \& comparées par des tests du \(\chi^2\). Des
graphiques pourront être réalisés pour les variables importantes.

\hypertarget{calcul-du-nombre-de-cas}{%
\section{Calcul du nombre de cas}\label{calcul-du-nombre-de-cas}}

Le critère de base est l'évolution du Z-score entre la naissance \& 12
mois. Si on estime que la différence entre les deux groupe sera
d'environ \(\num{0.5}\) , il faudrait \textbf{70 cas} par groupe (risque
\(\alpha\) 0,05 \& \(\beta\) 0,2).

\hypertarget{description}{%
\section{Description}\label{description}}

Les données démographiques ainsi que le bilan à la naissance seront
présentés sur un tableau en comparant les deux groupes pour rechercher
d'éventuels biais.

Le décompte des données manquantes variable par variable sera réalisé \&
présenté par un tableau ou un graphique. Une analyse de corrélations
portant sur tous les items sera réalisée. À la suite de ces contrôles
des variables pourront être exclues de la suite de l'analyse (trop de
donnée manquantes ou variables trop corrélées) avec l'accord du
promoteur.

Une analyse factorielle sera tentée (Analyse en Correspondances
Multiples) après imputation des données manquantes avec classification
des cas si possible.

\hypertarget{crituxe8re-principal}{%
\section{Critère principal}\label{crituxe8re-principal}}

\emph{Comparer la croissance pondérale des nouveau-nés prématurés
tardifs (\texttt{PT}) pris en charge en Unité Kangourou (\texttt{UK})
versus ceux pris en charge en néonatologie.}

L'évolution du Z-score pour le poids sera comparée entre les deux
groupes par un test du t de Student si les conditions de normalité sont
remplies sinon par un test de Wilcoxon. Plusieurs graphiques seront
dessinés :

\begin{itemize}
\tightlist
\item
  Comparaison simple (Box-plot ou Violin graph)
\item
  Évolution cas par cas (graphique en fagot)
\item
  Graphique en réseau d'évolution après regroupement en classe des cas.
\end{itemize}

Un lien entre les données démographique ou à la naissance \& l'évolution
sera recherché. Une analyse multivariée par régression linéaire sera
ensuite réalisée en incorporant dans un premier temps toutes les
variables ayant une p-value \textless{} 0,20 sur l'analyse monovariée.
Une recherche du meilleur modèle sera ensuite réalisé par un
step-by-step descendant. Pour cette détermination du meilleur modèle de
régression logistique on utilisera les données après imputation des
données manquantes. Par contre, une fois le modèle retenu, les calculs
présentés seront ceux réalisés sur les données réelles.

\hypertarget{technique}{%
\section{Technique}\label{technique}}

L'analyse statistique sera réalisée avec le logiciel
\textbf{R}\citep{rstat} \& divers packages en particulier
\texttt{baseph} \citep{baseph} \texttt{tidyverse} \citep{tidy} \&
\texttt{FactoMineR} \citep{facto}.


  \bibliography{stat.bib}


\end{document}
